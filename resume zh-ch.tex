% !TEX program = xelatex

\documentclass{resume}
\usepackage{zh_CN-Adobefonts_external} % Simplified Chinese Support using external fonts (./fonts/zh_CN-Adobe/)
%\usepackage{zh_CN-Adobefonts_internal} % Simplified Chinese Support using system fonts
\usepackage{linespacing_fix} % disable extra space before next section
\usepackage{cite}

\begin{document}
\pagenumbering{gobble} % suppress displaying page number

\name{杨朔柳}

\basicInfo{
  \email{visysl@outlook.com} \textperiodcentered\
  \phone{(+86) 155-571-80427} \textperiodcentered}

\section{\faGraduationCap\ 教育经历}
\datedsubsection{\textbf{浙江大学}}{2017 -- 2020}
\textit{硕士} 计算机科学与技术学院
\datedsubsection{\textbf{中南大学}}{2013 -- 2017}
\textit{学士} 航空航天学院

\section{\faUsers\ 工作经验}
\datedsubsection{\textbf{网易(杭州)网络技术有限公司} }{2020.4 -- 至今}
\role{游戏客户端开发}{}
\begin{itemize}
  \item 客户端
  \subsection{客户端游戏逻辑,UI编辑器的维护和开发。UI编辑器主要通过组件绑定,生成MVVM的代码框架,用来方便的实现业务逻辑。}

  \item GPU着色器,UI材质开发
  \subsection{实现相应的UI动画效果。}

  \item 开发和维护相应的工具链
  \subsection{对相应的本地化工具和编辑器工具做维护。负责游戏引擎插件的功能维护和开发。}

\end{itemize}

\section{\faCogs\ 工作技能}
\begin{itemize}[parsep=0.5ex]
  \item 编程语言: C/C++,Rust,Python,C\#,CUDA etc.
  \item 较为熟悉计算机图形学和渲染相关的知识。读研期间主要方向是科学计算可视化,针对科学计算产生的数据使用图形学的方式进行渲染,呈现给用户。
  \item 具有较为熟练的Qt GUI开发经验。
  \item 具有基本的操作系统相关的知识。
\end{itemize}

\section{\faUsers\ 项目}

\begin{itemize}

\item \textbf{Vulkan 教程} (\url{https://space.bilibili.com/12690590/channel/detail?cid=180474})

\begin{itemize}
  \item 面向Vulkan的初学者教程。Vulkan作为现代图形API,具有底层和复杂的特点,本视频通过手把手写代码的过程展示了Vulkan的基础特性。使初学者能快速掌握Vulkan的核心特性。
\end{itemize}


\item \textbf{大规模体数据渲染器} (\url{https://github.com/cad420/VolumeVisualizationGL})

\begin{itemize}
  \item
  \subsection{硕士论文项目。使用基于虚拟内存方式的工作集管理的大规模体数据绘制方法,为了在保证绘制质量的前提下减小I/O,加入了基于GPU视频编码解码以及混合分辨率技术。同时,利用现代图形API的多线程友好的特性,以多线程的方式进行GPU和内存之间的数据交互,能充分地利用I/O带宽,实现了大规模数据交互绘制,达到了单机上交互的大规模体数据可视化任务目标。}
\end{itemize}


\item \textbf{C++ 基础类库} (\url{https://github.com/cad420/VMCore}, \url{https://github.com/cad420/VMat},\\
 \url{https://github.com/cad420/VMUtils})

\begin{itemize}

\item 自己日常进行C++开发的基础库。包括智能指针,插件框架和内存管理模块。以及图形开发常用的基础运算。
\end{itemize}

\item \textbf{EMExplore} (\url{https://github.com/yslib/EMExplorer})

\subsection{
  读研期间的项目。开发了针对扫描电镜数据的进行图像处理,特征标记和标记管理,体渲染的一体化编辑器。
  主要是用于对扫描电镜数据的研究提供一个方便的自动化工作流。获取体术据之后,通过图像处理的方法对原始数据进行处理和增强,更容易使人判断出特征,
  然后通过画笔等工具进行手动或自动化的方法标记特征,并且能对特征进行管理。并且可以同时绘制出数据切片和体绘制和标记混合渲染的结果。
}

\end{itemize}

\section{\faUsers\ 出版物}

\begin{itemize}

\item \textbf{Xiangyang He, Shuoliu Yang, Yubo Tao, Haoran Dai, Hai Lin. Graph Convolutional Network-based Semi-supervised Feature Classification of Volumes. Journal of Visualization, 2021}
\item \textbf{Xiangyang He, Yubo Tao, Shuoliu Yang, Chuanchang Chen, Hai Lin. ScalarGCN: Scalar-value Association Analysis of Volumes based on Graph Convolutional Network. Journal of Visualization, 2021}

这两篇文章主要是利用图神经网络对体术据进行学习,然后利用训练波形对体术据进行特征分类,

\end{itemize}

\section{\faInfo\ 其他}
\begin{itemize}[parsep=0.5ex]
  \item 技术博客: \url{http://visysl.com}
  \item GitHub: \url{https://github.com/yslib}
\end{itemize}


\end{document}
